\section{Introduction}


	\subsection{Philosophie de test}
	\EPSFIGSCALE[1]{../comon/figures/testApercu.pdf}{Vue global de l'architecture du système}{archGlob}
	Les tests se feront sur trois niveaux, le premier niveau est celui des tests de fonctionnalités qui sont des tests fait par l'humain selon un procédés décrit dans le chapitre~\ref{tc}.  Le deuxième niveau est celui faits dans le code et qui sont reproductive automatiquement(Unit test).Le troisième niveau est celui des tests de fuite dans la mémoire(Memory leeks), qui consiste à observer la mémoire lors de l'utilisation(cas d'utilisation chapitre~\ref{tc} ) de l'application et de vérifier qu'aucune variable est stocké en mémoire pour toujours.  Il est important d'indiquer que les cas de test sont imaginés en même temps que la spécification, ce qui nous permet d'avoir un point de référence concernant les objectif à atteindre. Voir le \gls{SPMP} chapitre ''Processus technique'' pour plus d'information.\\
	 A la fin de chaque itération, un protocole de test est rédigé après avoir tester les nouvelles fonctionnalités et retester les anciennes fonctionnalités. Grâce à cette stratégie, on est sûr que les nouvelles fonctionnalités n'empêche pas le fonctionnement des anciennes et que le toute reste compatible.
\section{Organisation des tests }
	\subsection{Éléments à tester}
		\begin{itemize}
			\item Le bon fonctionnements des différents cas d'utilisations
			\item Des tests unitaires pour la partie logique métier.
			\item Des analyses de fuites dans la mémoire(Leek) doit être faite, vue que pour être visible sur l'appstore, une application ne doit pas contenir de Leek.
		\end{itemize}
	\subsection{Éléments à ne pas tester}
		\begin{itemize}
			\item La sécurité des web services
			\item La cohérence des résultats retourné par les web services.
		\end{itemize}
	\subsection{Outils de test et environnement}
			\todo{Complete}
\section{Cas de test \label{tc}}
	\subsection{Navigation}
				 \begin{longtable}{m{4cm}|p{10cm}|}
				 \textbf{ ID} & 1 \\
				 \hline \textbf{Déscirption} & Test que l'on peut bien naviguer d'un vue à l'autre sans erreurs\\
				 \hline \textbf{Déroulement} &
					 \begin{itemize}
						 \item Fermer complètement l'application si elle était ouverte
						 \item Ouvrir l'application
						 \item  Pour chaque vue, cliquer sur le logo pour ouvrir, voir si le résultat obtenu est cohérent, revenir au menu principal.
						 \item  Pour chaque vue, cliquer sur le logo pour ouvrir, modifier le contenu dans la vue, fermer l'application à l'aide du bouton menu de l'appareil, réouvrir l'application, vérifie que c'est toujours cette vue qui est visible et qu'aucune information n'a été perdu après la manœuvre. 
						 \item  Pour chaque vue, cliquer sur le logo pour ouvrir. faire 4 x une rotation de 90 degrés à l'appareil. Vérifier qu'après chaque rotation la vue est dans le bon sens et que les éléments sont affiché correctement. 
					 \end{itemize}
				 \\
			 \end{longtable} 
	\subsection{Paramétrer}
			 \begin{longtable}{m{4cm}|p{10cm}|}
			 \textbf{ ID} & 2 \\
			 \hline \textbf{Déscirption} & Test que l'on peut modifier les paramètres de l'application\\
			 \hline \textbf{Déroulement} &
				 \begin{itemize}
					 \item Fermer complètement l'application si elle était ouverte.
					 \item Ouvrir l'application.
					 \item  Ouvrir la fenêtre de paramètres.
					 \item  Pour chaque champs:
						 \begin{enumerate}
						 	\item Éditer la valeurs.
						 	\item fermer la fenêtre de paramètres.
						 	\item Réouvrir la fenêtre de paramètres.
						 	\item Vérifier que les valeurs sont bien celle saisi auparavant.
						 \end{enumerate}
					 \item Modifier tout les paramètres.
					 \item Fermer complètement l'application(À l'aide du gestionnaire d'application et non seulement à l'aide du bouton menu. )
					 \item Réouvrir l'application et être sûr que les modifications ont bien été enregistré.
				 \end{itemize}
			 \\
		 \end{longtable} 
		\subsection{Carte}
					 \begin{longtable}{m{4cm}|p{10cm}|}
					 \textbf{ ID} & 3 \\
					 \hline \textbf{Déscirption} & Test du bon fonctionnement de la carte.\\
					 \hline \textbf{Déroulement} &
						 \begin{itemize}
						  	\item Se rendre au campus de l'ESIB
						  	\item Se connecter à internet
							 \item Fermer complètement l'application si elle était ouverte.
							 \item Ouvrir l'application.
							 \item Ouvrir la fenêtre de la carte.
							 \item Presser le bouton Localiser moi et vérifier que l'endroit retourné et au bon emplacement.
							 \item Vérifier que l'application afficher des indicateurs sur les principaux immeubles du campus.
							 \item Se déplacer dans  le campus est vérifier que l'indicateur de position actuelle suit le déplacement.
							\item Saisir le nom d'une personne dans la bar de recherche, vérifier qu'on obtient en suivants les écrans un indicateurs concernant l'emplacement du bureau de cette personne.
							\item Saisir le nom d'une classe dans la bar de recherche et faire de même que l'étape précédente.
							\item Saisir le nom d'un bâtiment dans la bar de recherche et faire de même que l'étape précédente.
							\item Pressez sur le bouton de Navigation par élément.
							\begin{itemize}
								\item Choisir: Bâtiments 
								\item Choisir un bâtiments  spécifique et vérifier que son emplacement est affiché sur la carte. 
							\end{itemize}
							\item Se déconnecter d'Internet et recommencer les étapes précédentes. Les mêmes fonctionnalités doivent être visible
						 \end{itemize}
					 \\
				 \end{longtable} 
				 
		\subsection{News}
					 \begin{longtable}{m{4cm}|p{10cm}|}
					 \textbf{ ID} & 4 \\
					 \hline \textbf{Déscirption} & Test du bon fonctionnement de l'affichage des news.\\
					 \hline \textbf{Déroulement} &
						 \begin{itemize}
						  	\item Installer à neuf l'application
						  	\item Se connecter à internet
							 \item Fermer complètement l'application si elle était ouverte.
							 \item Ouvrir l'application.
							 \item Ouvrir la fenêtre des news.
							 \item Comparer le résultat avec celui de la page internet:\url{http://www.usj.edu.lb/}
							 \item Visualiser le détail des news et vérifier que le détail correspond à la news.
							\item Se déconnecter d'Internet et recommencer les étapes précédentes. Les mêmes fonctionnalités doivent être visible
						 \end{itemize}
					 \\
				 \end{longtable} 
				 
				 
				 
\section{Protocole de test}
		\subsection{Environnement de test}
		\todo{Write exactly version of soft and hardware of computer and devices}
		\subsection{Protocole de test 1}
		\textbf{Version testé:} 0.1 (\url{https://esibpad.googlecode.com/svn/tags/0.1}) \\
		\textbf{	Date du test :} 20/06/2011

		\subsubsection*{Cas de test : Navigation}
				 \begin{longtable}{m{4cm}|p{10cm}|}
				 \textbf{ ID} & 1 \\
				 \hline \textbf{Déscirption} & Test que l'on peut bien naviguer d'un vue à l'autre sans erreurs\\
				 \hline \textbf{Commentaires} &Il existe pour le moment qu'une seule page. \\
				 \hline Objectif  atteint & {\color{green} Complètement 100\% \CheckedBox } \\
				\hline Visa & Elias Medawar \\	
				 \\
			 \end{longtable} 
 		\subsubsection*{Cas de test : Paramétrer}
		 \begin{longtable}{m{4cm}|p{10cm}|}
		 \textbf{ ID} & 2 \\
		 \hline \textbf{Déscirption} & Test que l'on peut modifier les paramètres de l'application\\
		 \hline \textbf{Commentaires} & 
		 	 	 \begin{enumerate}
				  		\item La fonction retenir n'est pas encore implémenté correctement, les valeurs sont de toutes façon enregistré.
				  		\item La validité des champs n'est pas implémenté, les valeurs peuvent être incohérente.
				  		\item Les valeurs des champs ''Retenir et carte'' ne sont enregistrés qu'en cas de modification d'un autre champ de type texte
				  	\end{enumerate} \\
 				\hline Objectif atteint &  {\color{red}partiellement 75\% \XBox} \\
 				\hline Visa & Elias Medawar 	\\
		 \\
		 \end{longtable} 
		 \subsubsection*{Test unitaire}
		 \begin{lstlisting}[language=C,caption = Log des test unitaires]
Test Suite 'ESIB_PADTests' started at 2011-06-20 06:33:02 +0000
Test Case '-[ESIB_PADTests testSettings]' started.
 Testing the settings DAO
Test Case '-[ESIB_PADTests testSettings]' passed (0.003 seconds).
Test Suite 'ESIB_PADTests' finished at 2011-06-20 06:33:02 +0000.
Executed 1 test, with 0 failures (0 unexpected) in 0.003 (0.003) seconds
		 \end{lstlisting}
		Objectif atteint : {\color{green}Complètement 100 \% \CheckedBox}
		 \subsubsection*{Test de fuite dans la mémoire}
		 		 \EPSFIGTEXTWIDTH{../comon/figures/leeks_graph1.png}{Résultat de l'analyse des Leeks à l'aide d'Xcode}{leeks0.1}
		 Objectif atteint : {\color{green}Complètement 100 \% \CheckedBox}\\
		 On peut voir que le code ne contient aucune fuite de mémoire.
		 
		 
		\subsection{Protocole de test 2}
		 		\textbf{Version testé:} 0.2 (\url{https://github.com/eia-fr/ESIB_PAD/tree/0.2}) \\
		 		\textbf{	Date du test :} 05/07/2011
		 
		 		\subsubsection*{Cas de test : Navigation}
		 				 \begin{longtable}{m{4cm}|p{10cm}|}
		 				 \textbf{ ID} & 1 \\
		 				 \hline \textbf{Déscirption} & Test que l'on peut bien naviguer d'un vue à l'autre sans erreurs\\
		 				 \hline \textbf{Commentaires} &Lors du chargement des informations depuis internet, le logo loading n'est pas centrer quand l'iphone est en paysage. \\
		 				 \hline Objectif  atteint & {\color{green} Complètement 100\% \CheckedBox } \\
		 				\hline Visa & Elias Medawar \\	
		 				 \\
		 			 \end{longtable} 
		  		\subsubsection*{Cas de test : Paramétrer}
		 		 \begin{longtable}{m{4cm}|p{10cm}|}
		 		 \textbf{ ID} & 2 \\
		 		 \hline \textbf{Déscirption} & Test que l'on peut modifier les paramètres de l'application\\
		 		 \hline \textbf{Commentaires} & 
		 		 	 	 \begin{enumerate}
		 				  		\item La fonction retenir n'est pas encore implémenté correctement, les valeurs sont de toutes façon enregistré.
		 				  		\item La validité des champs est validé seulement au moment que les utiliser
		 				  		\item La carte est toujours en mode satellite.
		 				  	\end{enumerate} \\
		  				\hline Objectif atteint &  {\color{red}partiellement 85\% \XBox} \\
		  				\hline Visa & Elias Medawar 	\\
		 		 \\
		 		  \end{longtable} 		 		 
		 		 \subsubsection*{Cas de test : Carte}
		 		 		 \begin{longtable}{m{4cm}|p{10cm}|}
		 		 		 \textbf{ ID} & 3 \\
		 		 		 \hline \textbf{Déscirption} &  Test du bon fonctionnement de la carte.\\
		 		 		 \hline \textbf{Commentaires} &  
		 		 		 	 	 \begin{enumerate}
	 		 		 		 	 		\item La position de l'utilisateur est de toute façon affiché.
	 		 							\item Les coordonnées latitude longitude sont inversé pour les batiments du campus CTS
	 		 							\item {\color{red}L'application crache quand un campus n'as pas de bâtiment à afficher}.
	 		 		 		 	\end{enumerate} \\
	 		 		 		  				\hline Objectif atteint & {\color{orange} Partiellement 95\% \XBox } \\
	 		 		 		  				\hline Visa & Elias Medawar 	\\
		 		 		 \\
		 		 \end{longtable} 
		 		 \subsubsection*{Test unitaire}
		 		 \begin{lstlisting}[language=C,caption = Log des test unitaires]
Test Suite 'ESIB_PAD_SOURCESTests' started at 2011-07-05 07:37:38 +0000
Test Case '-[ESIB_PAD_SOURCESTests testCarte]' started.
 Testing the Campus DAO: You must uninstall the application before using this test
 Loading async the campus data from internet
 Waiting 30 sec for disabling the internet connection
 Getting campus data from cache?
 Comparing loacl and distant data
Test Case '-[ESIB_PAD_SOURCESTests testCarte]' passed (90.071 seconds).
Test Case '-[ESIB_PAD_SOURCESTests testSettings]' started.
 Testing the settings DAO
Test Case '-[ESIB_PAD_SOURCESTests testSettings]' passed (0.004 seconds).
Test Suite 'ESIB_PAD_SOURCESTests' finished at 2011-07-05 07:39:08 +0000.
Executed 2 tests, with 0 failures (0 unexpected) in 90.075 (90.077) seconds
		 		 \end{lstlisting}
		 		Objectif atteint : {\color{green}Complètement 100 \% \CheckedBox}\\
		 		\\
		 		Tester des appels de méthodes asynchrone n'est pas une tâche évidente. Pour parvenir à faire cet opération on attend un laps de temps de 30 secondes durant lequel l'application devrait en théorie avoir le temps de charger les données depuis internet et nous les transmettre. Pour recevoir les données, notre classe de test implémente le protocole  MapDisplayerDelegate. 
		 		 \subsubsection*{Test de fuite dans la mémoire}
		 		 	%\EPSFIGTEXTWIDTH{../comon/figures/leeks_graph1.png}{Résultat de l'analyse des Leeks à l'aide d'Xcode}{leeks0.1}
		 		 Objectif atteint : {\color{red}partiellement 50 \% \CheckedBox}\\
		 		 Un problème avec la classe NSPredicate crée des Leek, selon la théorie le code n'en contient pas. Mais l'outil de mesure en détecte, des recherches plus approfondi pour trouver une solution seront refait pour trouver un solution
		 
		\subsection{Protocole de test 3}
		 		\textbf{Version testé:} 0.3 (\url{https://esibpad.googlecode.com/svn/tags/0.1}) \\
		 		\textbf{	Date du test :} 05/12/2011
		 
		 		\subsubsection*{Cas de test : Navigation}
		 				 \begin{longtable}{m{4cm}|p{10cm}|}
		 				 \textbf{ ID} & 1 \\
		 				 \hline \textbf{Déscirption} & Test que l'on peut bien naviguer d'un vue à l'autre sans erreurs\\
		 				 \hline \textbf{Commentaires} &Lors du chargement des informations depuis internet, le logo loading n'est pas centrer quand l'iPhone		 est en paysage. \\
		 				 \hline Objectif  atteint & {\color{green} Complètement 100\% \CheckedBox } \\
		 				\hline Visa & Elias Medawar \\	
		 				 \\
		 			 \end{longtable} 
		  		\subsubsection*{Cas de test : Paramétrer}
		 		 \begin{longtable}{m{4cm}|p{10cm}|}
		 		 \textbf{ ID} & 2 \\
		 		 \hline \textbf{Déscirption} & Test que l'on peut modifier les paramètres de l'application\\
		 		 \hline \textbf{Commentaires} & 
		 		 	 	 \begin{enumerate}
		 				  		\item La fonction retenir n'est pas encore implémenté correctement, les valeurs sont de toutes façon enregistré.
		 				  		\item La validité des champs est validé seulement au moment que les utiliser
		 				  		\item La carte est toujours en mode satellite.
		 				  	\end{enumerate} \\
		  				\hline Objectif atteint &  {\color{red}partiellement 85\% \XBox} \\
		  				\hline Visa & Elias Medawar 	\\
		 		 \\
		 		  \end{longtable} 		 		 
		 		 \subsubsection*{Cas de test : Carte}
		 		 		 \begin{longtable}{m{4cm}|p{10cm}|}
		 		 		 \textbf{ ID} & 3 \\
		 		 		 \hline \textbf{Déscirption} &  Test du bon fonctionnement de la carte.\\
		 		 		 \hline \textbf{Commentaires} &  
		 		 		 	 	 \begin{enumerate}
	 		 		 		 	 		\item La position de l'utilisateur est de toute façon affiché.
	 		 							\item Les coordonnées latitude longitude sont inversé pour les batiments du campus CTS
	 		 							\item {\color{red}L'application crache quand un campus n'as pas de bâtiment à afficher}.
	 		 		 		 	\end{enumerate} \\
	 		 		 		  				\hline Objectif atteint & {\color{orange} Partiellement 95\% \XBox } \\
	 		 		 		  				\hline Visa & Elias Medawar 	\\
		 		 		 \\
		 		 \end{longtable} 
		 		 \subsubsection*{Test unitaire}
		 		 \begin{lstlisting}[language=C,caption = Log des test unitaires]
Test Suite 'ESIB_PAD_SOURCESTests' started at 2011-07-12 14:40:32 +0000
Test Case '-[ESIB_PAD_SOURCESTests testCarte]' started.
 Testing the Campus DAO: You must uninstall or reset cache of the application before testing
 Loading async the campus data from internet
 DATA recieved
Test Case '-[ESIB_PAD_SOURCESTests testCarte]' passed (60.564 seconds).
Test Case '-[ESIB_PAD_SOURCESTests testNews]' started.
 Testing the News DAO: You must uninstall or reset cache of the application before testing
 Loading async the news data from internet
 DATA recieved
Test Case '-[ESIB_PAD_SOURCESTests testNews]' passed (61.191 seconds).
Test Case '-[ESIB_PAD_SOURCESTests testSettings]' started.
Testing the settings DAO
Test Case '-[ESIB_PAD_SOURCESTests testSettings]' passed (0.073 seconds).
Test Suite 'ESIB_PAD_SOURCESTests' finished at 2011-07-12 14:42:34 +0000.
Executed 3 tests, with 0 failures (0 unexpected) in 121.828 (121.830) seconds
		 		 \end{lstlisting}
		 		Objectif atteint : {\color{green}Complètement 100 \% \CheckedBox}\\
		 		\\
		 		Le même principe de test que celui utilisé pour tester la carte a été mis en place.
		 		 \subsubsection*{Test de fuite dans la mémoire}
		 		 	%\EPSFIGTEXTWIDTH{../comon/figures/leeks_graph1.png}{Résultat de l'analyse des Leeks à l'aide d'Xcode}{leeks0.1}
		 		 Objectif atteint : {\color{red}partiellement 50 \% \CheckedBox}\\
		 		Le même problème que 