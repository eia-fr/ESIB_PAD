\section{Introduction}
	\subsection{But du document}
		Ce document décrit les variantes d'architecture étudié pour le projet ESIP@PAD, l'architecture finale choisie ainsi que le détail du design final du projet. A l'aide de ce document il est possible de comprendre le fonctionnement technique de l'ensemble du projet.
	\subsection{Aperçu du document}
		\todo{Describe this}
\section{Architecture du système }
	\subsection{Architecture choisie}	
		 \EPSFIGTEXTWIDTH{../comon/figures/GlobalArchitect.pdf}{Vue global de l'architecture du système}{archGlob}
		 Ce diagramme( Figure~\ref{archGlob}) nous donne un aperçu de l'architecture du système. On peut voir que la partie commune à l'iPhone et l'iPad a été regroupé en une seule partie, pour ainsi éviter la redondance de code.
		\subsection{Discussion des alternatives d'architectures}
	\subsection{Composants du système}
		\EPSFIGSCALE[0.7]{../comon/figures/ComposantSystem.pdf}{Diagrammes de composant du système}{archGlob}
		L'application est découpé en composant pour ainsi permettre de bien séparer les tâches que l'on désire offrir, facilité la réutilisation de partie de l'application et rendre les tests plus efficace vu que l'on ce concentre sur une partie et non pas un toute.
		
		\todo{Describe each component}

\section{Conception et Implémentation des composants}
	\subsection{MainView}
		\subsubsection*{Diagramme de séquence}
			\EPSFIGTEXTWIDTH{../comon/figures/seqNavig.pdf}{Diagramme de séquence du principe de la navigation}{seqNavig}
			Le diagramme de séquence est valable pour les deux appareil la seul différence est que sur l'IPad la vue chargé  ne cachera pas l'écran entier mais rien qu'une partie de l'écran.
		\subsubsection*{Diagramme de classe}
			 \EPSFIGSCALE[0.7]{../comon/figures/ClasMainViewIPhone.pdf}{Diagramme de classe du composant MainView}{classNavig}
		\subsubsection*{Discussion}
		\todo{Discussion}
	\subsection{Settings}
		\subsubsection*{Diagramme de séquence}
			\EPSFIGTEXTWIDTH{../comon/figures/seqSettings.pdf}{Diagramme de séquence concernant la modification des paramètres}{seqSet}
			Le diagramme de séquence est valable pour les deux appareil la seul différence est que sur l'IPad la vue chargé  ne cachera pas l'écran entier mais rien qu'une partie de l'écran.
		\subsubsection*{Diagramme de classe}
			 	\EPSFIGTEXTWIDTH{../comon/figures/ClassSettings.pdf}{Diagramme de classe du composant MainView sur iPhone}{classNavig}
		\subsubsection*{Discussion}
		\todo{Discussion}
		
	\subsection{Map}
		\subsubsection*{Diagramme de séquence}
			\EPSFIGTEXTWIDTH{../comon/figures/seqCarte.pdf}{Diagramme de séquence concernant l'affichage de la carte}{seqCarte}
			\EPSFIGTEXTWIDTH{../comon/figures/seqCarteSearch.pdf}{Diagramme de séquence concernant la recherche d'un élément sur la carte}{seqCarte}
			Le diagramme de séquence est valable pour les deux appareil la seul différence est que sur l'IPad la vue chargé  ne cachera pas l'écran entier mais rien qu'une partie de l'écran.
		\subsubsection*{Diagramme de classe}
			 	\EPSFIGTEXTWIDTH{../comon/figures/classCarte.pdf}{Diagramme de classe du composant Map}{classNavig}
		\subsubsection*{Discussion}
		\todo{Discussion}
