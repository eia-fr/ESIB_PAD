% % Corrected by carline
\section{Introduction}
	Ce document décrit brièvement les résultats obtenus à la fin de la deuxième itération de la phase de réalisation du projet ESIB@PAD. Le but de ce document est aussi de faire une brève analyse des connaissances acquises lors de cette itération et de savoir comment les exploiter au mieux pour les itérations suivantes. 
\section{Rappel des objectifs de la release }
		 \begin{longtable}{|c|l|p{10cm}|}
			 \hline  \textbf{Nom} & \textbf{Délais}  & \textbf{But à atteindre}  \\ 
			 \endfirsthead
			  \multicolumn{3}{|r|}{{suite de la page précédente}} \\ \hline
			\hline  \textbf{Nom} & \textbf{Délais}  & \textbf{But à atteindre}  \\ 
			 \endhead
			  \multicolumn{3}{|r|}{{Suite à la page suivante}} \\ \hline
			 \endfoot
			 \endlastfoot
		 	 \hline  Release 0.2 & 01.07.11 & 
			 \begin{itemize}
			 		 	 		\item Afficher la carte du campus.
			 		 	 		\begin{enumerate}[a)]
			 	 	 				\item La position actuelle de l'utilisateur sera détectée à l'aide du \gls{GPS} de l'appareil et affichée sur la carte.
			 	 	 				\item L'utilisateur peut, à l'aide de la fonction <<chercher>> : trouver l'emplacement d'un cours, le bureau d'une personne ou le lieu d'un événement.
			 	 	 				\item Les informations de la carte sont enregistrées sur le serveur et peuvent être mises à jour à tout moment. Un système de cache évite de recharger la carte à chaque visite.
			 	 	 			\end{enumerate}
			 		 	\end{itemize}   \\ 
			 \hline 
		  \end{longtable} 
		  
\section{Résultats obtenus }
	 \EPSFIGSCALE[0.6]{../comon/figures/pSMap.png}{Affichage de l'emplacement de tout les campus sur iPad}{pSMap}
	 \EPSFIGSCALE[0.6]{../comon/figures/psChoixcampus.png}{Choix du campus iPad}{psChoixcampus}
	\EPSFIGSCALE[0.3]{../comon/figures/psBatprincip.png}{Une foie que le campus est choisit, ces principaux bâtiment sont affiché ainsi que la position actuelle de l'utilisateur }{psBatprincip}
	 \EPSFIGSCALE[0.6]{../comon/figures/psChoixPersonn.png}{Liste des personnes qui ont un bureau(avec des coordonnées gps) pour le campus choisit}{psChoixPersonn} 
	 Pour le moment peut de personnes ont saisi l'emplacement de leur bureaux et de ce faite il y a peu de données.\\
	 Les données concernant les utilisateurs sont stockés en local dans une base de données pour ainsi avoir l'information en cache et minimiser la communication entre les appareils et le serveurs. En même l'utilisation du cache nous permet un accès à l'information sans être dépendant de la connexion internet.
 	 \EPSFIGSCALE[0.6]{../comon/figures/psBatSearch.png}{Tout comme la liste des personnes, la liste des salles permet de trouver une salles spécifique.  }{psBatSearch} 
 	 \EPSFIGSCALE[0.6]{../comon/figures/psBatDisp.png}{Lorsqu'on clique sur un bâtiment ou une personne parmi les choix proposés, le sujet choisit est affiché sur la carte.  }{psBatDisp} 
 	 \EPSFIGTEXTWIDTH{../comon/figures/psMapIPhone.png}{Les mêmes possibilités et fonctionnalités sont disponible sur l'iPhone  }{psMapIPhone} 

 	
\section{Analyse de la planification}
	D'après la planification, on avait prévu 2 semaines pour atteindre les buts de cette itération. Il a fallut qu'une semaine et demie pour le faire. Ce qui fait qu'on a rattrapé le retard pris dans la première itération.

\section{Décision concernant la prochaine itération}
\begin{itemize}
	\item Pour gagner du temps essayer de trouver des solutions compatible directement avec les deux appareils(iPhone et iPad), pour éviter de faire 2 fois le  travail.
	\item Ne pas modifier la planification globale.
\end{itemize}

\section{Conclusion}
Cette $2^\textrm{ ème}$ itération c'est bien déroulé. Des outils tel que la mise en cache d'informations on été développé et pourront être réutilisés dans les prochaines étapes.