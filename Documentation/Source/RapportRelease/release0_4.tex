% % Corrected by carline
\section{Introduction}
	Ce document décrit brièvement les résultats obtenus à la fin de la cinquième itération de la phase de réalisation du projet ESIB@PAD. Le but de ce document est aussi de faire une brève analyse des connaissances acquises lors de cette itération et de savoir comment les exploiter au mieux pour les itérations suivantes. 
\section{Rappel des objectifs de la release }
		 \begin{longtable}{|c|l|p{10cm}|}
			 \hline  \textbf{Nom} & \textbf{Délais}  & \textbf{But à atteindre}  \\ 
			 \endfirsthead
			  \multicolumn{3}{|r|}{{suite de la page précédente}} \\ \hline
			\hline  \textbf{Nom} & \textbf{Délais}  & \textbf{But à atteindre}  \\ 
			 \endhead
			  \multicolumn{3}{|r|}{{Suite à la page suivante}} \\ \hline
			 \endfoot
			 \endlastfoot
			 
		\hline  Release 0.4 & 22.07.11  & 
				\begin{itemize}
 		 	 		\item Permettre l'accès à l'annuaire de l'université. 
	 		 	 	\begin{enumerate}[a)]
		 	 				\item Quand on clique sur un numéro de téléphone, l'appel est lancé.
		 	 				\item Quand on clique sur une adresse mail, la fenêtre d'envoi de mail de l'appareil est ouverte.
		 	 			\end{enumerate}
 		 	\end{itemize}   \\
			 \hline 
		  \end{longtable} 
		L'appel se fait uniquement si on utilise un iPhone. 
\section{Résultats obtenus }
	 \EPSFIGTEXTWIDTH{../comon/figures/WierframeIPhoneDirectory}{Affichage de l'annuaire sur l'iPhone}{WierframeIPhoneDirectory}
	 \EPSFIGTEXTWIDTH{../comon/figures/WierframeIPadDirectory}{Affichage de l'annuaire sur l'iPad}{WierframeIPadDirectory}

\section{Analyse de la planification}
Les objectifs ont été atteints avec un retard de 3-4 jours. Ce retard est dû au temps nécessaire au développement de toutes les interfaces graphiques à double.

\section{Décision concernant la prochaine itération}
Essayer de rattraper le retard pris lors de cette itération pour ne pas être pris de cours pour la fin de projet.

\section{Conclusion}
Cette $4^\textrm{ ème}$ itération s'est bien déroulée. Avec une interface graphique personnalisée telle que celle-ci, le temps pour produire une application compatible avec les 2 appareils est quasi le double de celui nécessaire pour une application compatible avec 1 seul appareil.