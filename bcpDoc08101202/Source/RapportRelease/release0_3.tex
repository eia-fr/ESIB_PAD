% % Corrected by carline
\section{Introduction}
	Ce document décrit brièvement les résultats obtenus à la fin de la troisième itération de la phase de réalisation du projet ESIB@PAD. Le but de ce document est aussi de faire une brève analyse des connaissances acquises lors de cette itération et de savoir comment les exploiter au mieux pour les itérations suivantes. 
\section{Rappel des objectifs de la release }
		 \begin{longtable}{|c|l|p{10cm}|}
			 \hline  \textbf{Nom} & \textbf{Délais}  & \textbf{But à atteindre}  \\ 
			 \endfirsthead
			  \multicolumn{3}{|r|}{{suite de la page précédente}} \\ \hline
			\hline  \textbf{Nom} & \textbf{Délais}  & \textbf{But à atteindre}  \\ 
			 \endhead
			  \multicolumn{3}{|r|}{{Suite à la page suivante}} \\ \hline
			 \endfoot
			 \endlastfoot
			 \hline  Release 0.3 & 13.07.11  & 
		 		 	\begin{itemize}
		 		 	 		\item Permettre de consulter les nouvelles du campus.
			 		 	 	\begin{enumerate}[a)]
			 		 	 			\item \sout{Si une nouvelle est liée à un lieu, permettre de l'afficher facilement sur la carte.}
			 		 	 		\end{enumerate}
		 		 	\end{itemize}   \\
			 \hline 
		  \end{longtable} 
		  L'objectif : Si une nouvelle est liée à un lieu, permettre de l'afficher facilement sur la carte, a été abandonné car la base de donnée ne contient pas les données de localisation d'une news.
		   
\section{Résultats obtenus }
	 \EPSFIGSCALE[0.9]{../comon/figures/psNewsIPho}{Affichage des news sur l'iPhone}{psNewsIPho}
	 \EPSFIGSCALE[0.9]{../comon/figures/psNewsIPa}{Affichage des news sur l'iPad}{psNewsIPa}

 	
\section{Analyse de la planification}
  Il a fallu moins de temps que prévu pour réaliser cette itération. Ce qui fait que  l'itération se termine avec 1 jour d'avance.

\section{Décision concernant la prochaine itération}
\begin{itemize}
	\item Si la tendance d'avance continue, modifier la planification pour rajouter  un objectif facultatif.
\end{itemize}

\section{Conclusion}
Cette $3^\textrm{ ème}$ itération s'est bien déroulé. Des problèmes de connexions(time-out ou coupure de connexion) internet ont été rencontré lors des commit du code source sur le serveur SVN ainsi que lors de l'upload de la documentation sur le site du projet. Ces problèmes ont pu être contournés avec le changement de serveur qui est désormais un serveur Git.La technologie Git est un peu plus rapide, ce qui augmente la probabilité de parvenir à transmettre les données sans coupures. Malgré cette astuce , les commits ainsi que la transmission des documents sur internet reste un problème à résoudre pour le moment.