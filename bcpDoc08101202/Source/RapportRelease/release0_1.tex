% % Corrected by carline
\section{Introduction}
	Ce document décrit brièvement les résultats obtenus à la fin de la première itération de la phase de réalisation du projet ESIB@PAD. Le but de ce document est aussi de faire une brève analyse des connaissances acquises lors de cette itération et de savoir comment les exploiter au mieux pour les itérations suivantes. \\ Cette première itération a permis de suivre complètement le processus technique  choisi dans le \gls{SPMP} et de voir le rythme auquel le travail avance. 
\section{Rappel des objectifs de la release }
		 \begin{longtable}{|c|l|p{10cm}|}
			 \hline  \textbf{Nom} & \textbf{Délais}  & \textbf{But à atteindre}  \\ 
			 \endfirsthead
			  \multicolumn{3}{|r|}{{suite de la page précédente}} \\ \hline
			\hline  \textbf{Nom} & \textbf{Délais}  & \textbf{But à atteindre}  \\ 
			 \endhead
			  \multicolumn{3}{|r|}{{Suite à la page suivante}} \\ \hline
			 \endfoot
			 \endlastfoot
		 	 \hline  Release 0.1 & 17.06.11  & 
			 	\begin{itemize}
			 	 		\item La page d'accueil de l'application avec les différents menus est réalisée.
			 			\item La page de paramètres de l'application est réalisée.
			 	\end{itemize}   \\ 
			 \hline 
		  \end{longtable} 
		  
\section{Résultats obtenus }
	 \EPSFIGSCALE[0.6]{../comon/figures/psIPaAccueil.png}{Interface du programme sur l'iPAD  (Les icônes sont des icônes de menu pour exemple)}{psIPaAccueil}
	 Sur la Figure~\ref{psIPaAccueil} dans la {\color{red} zone A} , se trouve le menu de l'application qui est généré automatiquement depuis un fichier XML de paramètres. La {\color{green} zone B} est la zone de travail correspondant au menu sélectionné actuellement.
	 \EPSFIGSCALE[0.6]{../comon/figures/psIPhAcueilLand.png}{Page de paramétrage sur iPhone (Les icônes sont des icônes de menu pour exemple)  }{psIPhAcueilLand}
	 Sur la Figure~\ref{psIPhAcueilLand} se trouve le menu de l'application qui est lui aussi généré automatiquement depuis un fichier XML de paramètres. 	
 	\EPSFIGSCALE[0.6]{../comon/figures/psIPhSettings.png}{Page de paramétrage sur iPhone }{psIPhSettings}
 	
\section{Analyse de la planification}
	Dans la planification initiale, il était prévu de finir cette itération le vendredi 17.06. Mais les objectifs n'ont été atteints que le mardi 21.06. C'est en grande partie le temps nécessaire pour documenter qui a été sous-estimé. M.Wuergler ainsi que les responsables internes m'ont fait part de leurs sentiments d'une sur-estimation des  objectifs atteignable dans le temps à disposition. Nous avons discuté ce point avec M.Mezher il a dit qu'en cas de retard, l'objectif pour l'affichage des notes peut être mis de côté.

\section{Décision concernant la prochaine itération}
\begin{itemize}
	\item Prévoir plus de temps pour la documentation
	\item Garder la planification globale  actuelle et voir à la fin de la prochaine itération si il faut diminuer les objectifs. 
\end{itemize}

\section{Conclusion}
Malgré le petit retard de 2 jours, on peut considérer que les buts sont atteint et que le projet est sur une bonne voie. Cette itération a permis d'acquérir une bonne base de la programmation iOS qui sera utile pour la suite.	