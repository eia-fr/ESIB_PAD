% % Corrected by carline
\section{Introduction}
	Ce document décrit brièvement les résultats obtenus à la fin de la cinquième itération de la phase de réalisation du projet ESIB@PAD. Le but de ce document est aussi de faire une brève analyse des connaissances acquises lors de cette itération et de savoir comment les exploiter au mieux pour les itérations suivantes. 
\section{Rappel des objectifs de la release }
		 \begin{longtable}{|c|l|p{10cm}|}
			 \hline  \textbf{Nom} & \textbf{Délais}  & \textbf{But à atteindre}  \\ 
			 \endfirsthead
			  \multicolumn{3}{|r|}{{suite de la page précédente}} \\ \hline
			\hline  \textbf{Nom} & \textbf{Délais}  & \textbf{But à atteindre}  \\ 
			 \endhead
			  \multicolumn{3}{|r|}{{Suite à la page suivante}} \\ \hline
			 \endfoot
			 \endlastfoot
			 
\hline  Release 0.5 & 03.08.11  & 
 		 
 	 		 	\begin{itemize}
 	 		 	 		\item Permettre aux professeurs et aux étudiants d'afficher leurs horaires.
 		 		 	 	\begin{enumerate}[a)]
 		 		 	 			\item Quand on clique sur un cours, l'emplacement de ce dernier est affiché sur la carte.
 		 		 	 			\item L'utilisateur peut sauvegarder son horaire sur l'appareil pour un accès offline.
 		 		 	 		\end{enumerate}
 	 		 	\end{itemize}   \\ 
			 \hline 
		  \end{longtable} 
		L'horaire est de toute façon  sauvegardé sur l'appareil pour un accès offline
		   
\section{Résultats obtenus }
	 \EPSFIGTEXTWIDTH{../comon/figures/wireframeHorraireiPhone}{Affichage de l'horaire sur l'iPhone}{wireframeHorraireiPhone}
	 \EPSFIGTEXTWIDTH{../comon/figures/wireframeHorraireiPad}{Affichage de l'horaire sur l'iPad}{wireframeHorraireiPhone}

 	
\section{Analyse de la planification}
Les objectifs ont été atteints dans les temps. Les tests de cette release sont encore incomplets et doivent être améliorés. 

\section{Décision concernant la prochaine itération}
Profiter que l'objectif de la prochaine itération soit peu conséquent pour compléter les tests de la release 05.

\section{Conclusion}
Cette $5^\textrm{ ème}$ itération s'est bien déroulée.Les données sont uniquement accessible depuis les web services locaux car le service informatique de l'USJ n'a pas encore intégré cette opération sur le serveur. 